\textbf{Powtórka (z wykładu 10).}
\begin{itemize}
\item Szukamy $w_n^\ast\in \Pi_n$ minimalizującego błąd w normie dyskretnej:
$$
\|f-w_n^\ast\|_2=\min_{w\in\Pi_n}\|f-w\|_2.
$$
\item W bazie ortogonalnej $\{P_k\}$ mamy postać
$$
w_n^\ast(x)=\sum_{k=0}^{n} a_k P_k(x),
\qquad
 a_k=\frac{(f,P_k)_N}{(P_k,P_k)_N}.
$$
\item Współczynniki wyznaczamy projekcyjnie, jak w dyskretnym rozwinięciu Fouriera.
\end{itemize}

\section{Całki nieoznaczone i oznaczone -- przypomnienie}

Dla pochodnej odwrotnej (funkcji pierwotnej):
$$
\int f(x)\,dx = F(x)+C,\qquad F'(x)=f(x).
$$

Przykłady:
$$
\int x^n\,dx=\frac{x^{n+1}}{n+1}+C\quad (n\neq -1),
\qquad
\int \frac{dx}{x}=\ln|x|+C,
$$
$$
\int e^x\,dx=e^x+C,
\qquad
\int \cos x\,dx=\sin x+C,
\qquad
\int \frac{dx}{x^2+1}=\arctan x+C.
$$

Nie każda całka ma postać elementarną, np.
$$
\int e^{-x^2}\,dx,
\qquad
\int \frac{\sin x}{x}\,dx.
$$

Całka oznaczona:
$$
\int_a^b f(x)\,dx=F(b)-F(a),\qquad F'=f,
$$
co interpretujemy jako pole obszaru zorientowane pomiędzy wykresem i osią $OX$.

\begin{center}
\begin{tikzpicture}[x=1cm,y=1cm,>=Latex]
  \draw[->] (-0.2,0) -- (6.6,0);
  \draw[->] (0,-1.4) -- (0,2.0);
  \draw[thick] plot[smooth] coordinates {(0.4,0.2) (1.2,1.0) (2.1,0.4) (2.9,-0.8) (3.8,-0.2) (4.6,1.1) (5.7,1.4)};
  \draw[dashed] (0.4,0)--(0.4,0.2);
  \draw[dashed] (5.7,0)--(5.7,1.4);
  \node[below] at (0.4,0) {$a$};
  \node[below] at (5.7,0) {$b$};
  \node[right] at (5.8,1.4) {$f(x)$};
\end{tikzpicture}
\end{center}

\section{Idea całkowania numerycznego}

Jeśli $f$ jest trudna do scałkowania analitycznie, szukamy funkcji $g$ dobrze przybliżającej $f$ na $[a,b]$, takiej że
$$
\int_a^b f(x)\,dx\approx\int_a^b g(x)\,dx,
$$
a prawa strona daje się policzyć jawnie.

\section{Kwadratury liniowe}

\textbf{Definicja.}
Kwadraturą liniową nazywamy wyrażenie
$$
Q_n(f):=\sum_{k=0}^{n} A_k^{(n)}\, f\big(x_k^{(n)}\big),
$$
gdzie $x_0^{(n)},\dots,x_n^{(n)}$ to węzły, a $A_0^{(n)},\dots,A_n^{(n)}$ to współczynniki kwadratury.

\textbf{Zadanie.}
Dobrać węzły i współczynniki tak, aby
$$
\int_a^b f(x)\,dx\approx Q_n(f)
$$
dla możliwie szerokiej klasy funkcji $f$.

Mamy równanie z resztą:
$$
\int_a^b f(x)\,dx = Q_n(f)+R_n(f),
$$
gdzie $R_n(f)$ to błąd kwadratury.

\section{Rząd kwadratury}

\textbf{Definicja.}
Mówimy, że kwadratura liniowa $Q_n$ ma rząd $r\in\mathbb N$, gdy:
\begin{itemize}
\item dla każdego $w\in\Pi_{r-1}$ zachodzi
$$
\int_a^b w(x)\,dx = Q_n(w),
$$
\item istnieje $v\in\Pi_r$ takie, że
$$
\int_a^b v(x)\,dx \neq Q_n(v).
$$
\end{itemize}

\textbf{Uwaga.}
Kwadratury chcemy konstruować tak, aby ich rząd był możliwie wysoki.

\textbf{Twierdzenie.}
$$
\operatorname{rzad}(Q_n)\le 2n+2.
$$

\section{Kwadratura interpolacyjna}

Niech $L_n\in\Pi_n$ będzie wielomianem interpolacyjnym funkcji $f$ w węzłach $x_0,\dots,x_n$:
$$
L_n(x_k)=f(x_k),\qquad k=0,1,\dots,n.
$$
W postaci Lagrange'a:
$$
L_n(x)=\sum_{k=0}^{n} f(x_k)\,\lambda_k(x),
$$
$$
\lambda_k(x)=\prod_{\substack{i=0\\ i\ne k}}^{n}\frac{x-x_i}{x_k-x_i}.
$$

Po scałkowaniu:
$$
\int_a^b L_n(x)\,dx
=\sum_{k=0}^{n}\left(\int_a^b\lambda_k(x)\,dx\right)f(x_k)
=:\sum_{k=0}^{n}A_k f(x_k)=Q_n(f),
$$
czyli
$$
A_k=\int_a^b\lambda_k(x)\,dx.
$$

To jest właśnie kwadratura interpolacyjna.

\textbf{Przypomnienie (błąd interpolacji).}
Jeśli $f\in C^{n+1}[a,b]$, to dla $x\in[a,b]$:
$$
f(x)-L_n(x)=\frac{f^{(n+1)}(\eta_x)}{(n+1)!}\prod_{k=0}^{n}(x-x_k),
\qquad \eta_x\in(a,b).
$$
Stąd po scałkowaniu:
$$
\int_a^b f(x)\,dx-\int_a^b L_n(x)\,dx
=\frac{1}{(n+1)!}\int_a^b f^{(n+1)}(\eta_x)\prod_{k=0}^{n}(x-x_k)\,dx.
$$

\section{Przykład kwadratury interpolacyjnej}

Dla $[a,b]=[0,1]$, $n=4$, węzły:
$$
x_0=0,\quad x_1=\frac13,\quad x_2=\frac12,\quad x_3=\frac23,\quad x_4=1.
$$
Dla tej siatki:
$$
A_0=A_4=\frac{11}{120},\qquad
A_1=A_3=\frac{27}{40},\qquad
A_2=-\frac{8}{15}.
$$
Zatem
$$
Q_4(f)=\frac{11}{120}f(0)+\frac{27}{40}f\!\left(\frac13\right)-\frac{8}{15}f\!\left(\frac12\right)+\frac{27}{40}f\!\left(\frac23\right)+\frac{11}{120}f(1).
$$

Dla $f(x)=\sin(\pi x)$:
$$
\int_0^1\sin(\pi x)\,dx=\frac{2}{\pi}\approx 0.6366,
$$
$$
Q_4(f)=\frac{27\sqrt3}{40}-\frac{8}{15}\approx 0.6358.
$$

Dla $f(x)=e^{-x^2}$:
$$
\int_0^1 e^{-x^2}\,dx\approx 0.7468241,\qquad Q_4(f)\approx 0.746841.
$$

\section{Rząd a interpolacja}

\textbf{Twierdzenie.}
Kwadratura liniowa ma rząd co najmniej $n+1$ wtedy i tylko wtedy, gdy jest kwadraturą interpolacyjną.

W szczególności dla kwadratur interpolacyjnych:
$$
n+1\le \operatorname{rzad}(Q_n)\le 2n+2.
$$

Pytanie naturalne: jak dobrać węzły i współczynniki, by osiągnąć rząd dokładnie $2n+2$?
Odpowiedzią są kwadratury Gaussa.

\section{Kwadratury Newtona--Cotesa}

Bierzemy węzły równoodległe:
$$
x_k=a+kh,\qquad k=0,1,\dots,n,
\qquad h:=\frac{b-a}{n}.
$$

Dla tych węzłów współczynniki kwadratury interpolacyjnej można zapisać jawnie:
$$
A_k=\frac{(-1)^{n-k}}{k!(n-k)!}\,h\int_0^n\prod_{\substack{j=0\\ j\ne k}}^{n}(t-j)\,dt,
$$
a ponadto zachodzi symetria
$$
A_k=A_{n-k}.
$$

\textbf{Fakt (rząd kwadratur Newtona--Cotesa).}
\begin{itemize}
\item dla $n$ parzystych: $\operatorname{rzad}(Q_n^{NC})=n+2$,
\item dla $n$ nieparzystych: $\operatorname{rzad}(Q_n^{NC})=n+1$.
\end{itemize}

\section{Dwa klasyczne wzory NC}

\textbf{Wzór trapezów} ($n=1$):
$$
Q_1^{NC}(f)=\frac{b-a}{2}\,[f(a)+f(b)].
$$
Reszta ma rząd $O(h^3)$ (dla $h=b-a$).

\textbf{Wzór Simpsona} ($n=2$):
$$
Q_2^{NC}(f)=\frac{b-a}{6}\left[f(a)+4f\!\left(\frac{a+b}{2}\right)+f(b)\right],
$$
$$
R_2^{NC}(f)=\frac{1}{90}h^5 f^{(4)}(\alpha),\qquad \alpha\in(a,b),
$$
co daje rząd $O(h^5)$.

\textbf{Uwaga.}
Na zdjęciu ze wzorem trapezów współczynnik przy dokładnej postaci reszty jest nieczytelny; zachowałem pewną informację o rzędzie błędu.
