\textbf{Powtórka (z wykładu 11).}
\begin{itemize}
\item Kwadratura liniowa ma postać
$$
Q_n(f)=\sum_{k=0}^{n}A_k^{(n)}f\big(x_k^{(n)}\big).
$$
\item Rząd kwadratury:
$$
\operatorname{rzad}(Q_n)=r
\iff
\begin{cases}
Q_n(w)=\displaystyle\int_a^b w(x)\,dx & \forall w\in\Pi_{r-1},\\
\exists v\in\Pi_r:\ Q_n(v)\neq \displaystyle\int_a^b v(x)\,dx.
\end{cases}
$$
\item Dla kwadratur interpolacyjnych:
$$
n+1\le \operatorname{rzad}(Q_n)\le 2n+2.
$$
\end{itemize}

\section{Kwadratura interpolacyjna -- postać współczynników}

Dla zadanych węzłów $x_0,\dots,x_n$ i bazy Lagrange'a $\lambda_k$:
$$
Q_n(f)=\sum_{k=0}^{n}A_k f(x_k),
\qquad
A_k=\int_a^b \lambda_k(x)\,dx,
$$
$$
\lambda_k(x)=\prod_{\substack{i=0\\ i\ne k}}^{n}\frac{x-x_i}{x_k-x_i}.
$$

Dla węzłów równoodległych otrzymujemy kwadratury Newtona--Cotesa.

\section{Kwadratury złożone -- idea}

Dzielimy przedział całkowania na podprzedziały:
$$
a=t_0<t_1<\dots<t_n=b,
\qquad
\int_a^b f(x)\,dx=\sum_{k=0}^{n-1}\int_{t_k}^{t_{k+1}} f(x)\,dx.
$$

Na każdym małym przedziale stosujemy prostą kwadraturę (np. trapezy lub Simpsona), a następnie sumujemy.

\begin{center}
\begin{tikzpicture}[x=1.05cm,y=0.9cm,>=Latex]
  \draw[->] (-0.2,0) -- (7.4,0);
  \draw[->] (0,-0.2) -- (0,2.2);
  \draw[thick] plot[smooth] coordinates {(0.2,0.7) (1.0,1.3) (1.8,1.1) (2.7,0.5) (3.6,1.0) (4.5,1.5) (5.5,1.3) (6.8,1.7)};
  \foreach \x in {0.4,1.6,2.8,4.0,5.2,6.4} {
    \draw[dashed] (\x,0) -- (\x,2.0);
  }
  \node[below] at (0.4,0) {$t_0$};
  \node[below] at (6.4,0) {$t_n$};
\end{tikzpicture}
\end{center}

\section{Złożony wzór trapezów}

Niech siatka będzie równomierna:
$$
t_k=a+kh_n,\qquad h_n:=\frac{b-a}{n},\qquad k=0,1,\dots,n.
$$

Dla pojedynczego przedziału $[t_k,t_{k+1}]$:
$$
\int_{t_k}^{t_{k+1}} f(x)\,dx
=\frac{h_n}{2}\big(f(t_k)+f(t_{k+1})\big)-\frac{h_n^3}{12}f''(\eta_k),
\qquad \eta_k\in(t_k,t_{k+1}).
$$

Po zsumowaniu:
$$
\int_a^b f(x)\,dx=T_n(f)+R_n^T(f),
$$
$$
T_n(f):=h_n\left[\frac{f(a)+f(b)}{2}+\sum_{k=1}^{n-1}f(t_k)\right],
$$
$$
R_n^T(f)=-\frac{b-a}{12}h_n^2 f''(\xi),\qquad \xi\in(a,b),\quad f\in C^2[a,b].
$$

\textbf{Twierdzenie.}
Jeśli $f\in C[a,b]$, to
$$
\lim_{n\to\infty}T_n(f)=\int_a^b f(x)\,dx.
$$

\section{Złożony wzór Simpsona}

Dzielimy $[a,b]$ na parzystą liczbę podprzedziałów:
$$
n=2m,
\qquad
h_n=\frac{b-a}{n}.
$$

Dla każdej pary przedziałów $[t_{2k},t_{2k+2}]$:
$$
\int_{t_{2k}}^{t_{2k+2}} f(x)\,dx
=\frac{h_n}{3}\Big(f(t_{2k})+4f(t_{2k+1})+f(t_{2k+2})\Big)
+\frac{h_n^5}{90}f^{(4)}(\alpha_k),
$$
$$
\alpha_k\in(t_{2k},t_{2k+2}),\qquad f\in C^4[t_{2k},t_{2k+2}].
$$

Po zsumowaniu:
$$
\int_a^b f(x)\,dx=S_n(f)+R_n^S(f),
$$
$$
S_n(f):=\frac{h_n}{3}\left[f(t_0)+2\sum_{k=1}^{m-1}f(t_{2k})+4\sum_{k=1}^{m}f(t_{2k-1})+f(t_{2m})\right],
$$
$$
R_n^S(f)=\frac{a-b}{180}h_n^4 f^{(4)}(\alpha),\qquad \alpha\in(a,b),\quad f\in C^4[a,b].
$$

\textbf{Wniosek (rząd zbieżności).}
$$
R_n^T(f)=O(n^{-2}),\qquad R_n^S(f)=O(n^{-4}).
$$

\section{Metoda Romberga}

Bierzemy kolejne zagęszczenia siatki:
$$
n=2^k,\qquad h_k:=\frac{b-a}{2^k},\qquad x_i^{(k)}=a+ih_k\ (i=0,1,\dots,2^k).
$$

Pierwszy wiersz tablicy Romberga to wartości złożonego trapezu:
$$
T_{0k}:=T_{2^k}(f)=h_k\left[\frac{f(a)+f(b)}{2}+\sum_{i=1}^{2^k-1}f\big(x_i^{(k)}\big)\right].
$$

Następnie wykonujemy ekstrapolację Richardsona:
$$
T_{mk}:=\frac{4^m T_{m-1,k+1}-T_{m-1,k}}{4^m-1},
\qquad k=0,1,\dots,\quad m=1,2,\dots
$$

Własności (z tablicy Romberga):
\begin{itemize}
\item $T_{mk}=\displaystyle\int_a^b f(x)\,dx+O\!\left(h_k^{2m+2}\right)$,
\item każde $T_{mk}$ jest kwadraturą liniową,
\item ciąg na przekątnej $T_{00},T_{11},T_{22},\dots$ daje coraz wyższe rzędy.
\end{itemize}

\section{Informacja o kwadraturach Gaussa}

\textbf{Problem.}
Dla kwadratury
$$
Q_n(f)=\sum_{k=0}^{n}A_k^{(n)}f\big(x_k^{(n)}\big)
$$
chcemy dobrać węzły i współczynniki tak, aby
$$
\operatorname{rzad}(Q_n)=2n+2
$$
(maksymalny możliwy).

Taka kwadratura musi być interpolacyjna, więc
$$
A_k^{(n)}=\int_a^b\prod_{\substack{i=0\\ i\ne k}}^{n}\frac{x-x_i^{(n)}}{x_k^{(n)}-x_i^{(n)}}\,dx.
$$
Kluczowy problem to dobór węzłów $x_k^{(n)}$.

\textbf{Twierdzenie (Gauss--Legendre, przypadek $[a,b]=[-1,1]$).}
Węzły kwadratury rzędu $2n+2$ są miejscami zerowymi wielomianu Legendre'a $P_{n+1}$.

Wielomiany Legendre'a spełniają rekurencję:
$$
P_0(x)=1,\qquad P_1(x)=x,
$$
$$
P_k(x)=\frac{2k-1}{k}xP_{k-1}(x)-\frac{k-1}{k}P_{k-2}(x),
\qquad k=2,3,\dots
$$

\textbf{Uwagi.}
\begin{itemize}
\item Nie ma prostych wzorów jawnych na zera $P_{n+1}$.
\item W praktyce zera i współczynniki są tablicowane lub liczone numerycznie z dużą dokładnością.
\item Dzięki temu kwadratury Gaussa można stabilnie liczyć także dla dużych $n$.
\end{itemize}
