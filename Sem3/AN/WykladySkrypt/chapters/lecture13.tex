\textbf{Powtórka (z wykładu 12).}
\begin{itemize}
\item Złożony wzór trapezów: błąd rzędu $O(h^2)=O(n^{-2})$.
\item Złożony wzór Simpsona: błąd rzędu $O(h^4)=O(n^{-4})$.
\item Metoda Romberga: ekstrapolacja Richardsona dla kolejnych przybliżeń trapezowych.
\item Kwadratury Gaussa: maksymalny rząd $2n+2$ dla $n+1$ węzłów.
\end{itemize}

\section{Macierze -- przypomnienie}

Macierz rzeczywista $m\times n$:
$$
A=[a_{ij}]\in\mathbb R^{m\times n},
\qquad
b=\begin{bmatrix}b_1\\ \vdots \\ b_m\end{bmatrix}\in\mathbb R^m.
$$

Dla macierzy tego samego rozmiaru:
$$
A\pm B=[a_{ij}\pm b_{ij}],
\qquad
\lambda A=[\lambda a_{ij}],\ \lambda\in\mathbb R,
\qquad
A^T=[a_{ji}].
$$

Macierz jednostkowa:
$$
I_n=\operatorname{diag}(1,\dots,1)\in\mathbb R^{n\times n}.
$$

\section{Mnożenie macierzy i macierz odwrotna}

Jeśli
$$
A=[a_{ij}]\in\mathbb R^{m\times k},
\qquad
B=[b_{ij}]\in\mathbb R^{k\times n},
$$
to
$$
AB=[c_{ij}]\in\mathbb R^{m\times n},
\qquad
c_{ij}=\sum_{\ell=1}^{k} a_{i\ell}b_{\ell j}.
$$

Koszt klasycznego mnożenia: $O(n^3)$ (dla macierzy kwadratowych).
Mnożenie macierzy nie jest przemienne.

Dla macierzy kwadratowej $A\in\mathbb R^{n\times n}$:
$$
A^{-1}\ \text{spełnia}\ AA^{-1}=A^{-1}A=I_n.
$$

\section{Wyznacznik i odwracalność}

Wyznacznik:
$$
\det:\mathbb R^{n\times n}\to\mathbb R.
$$

Własności:
$$
\det(AB)=\det(A)\det(B),
\qquad
\det(A^T)=\det(A).
$$

\textbf{Twierdzenie.}
Macierz $A$ jest odwracalna wtedy i tylko wtedy, gdy
$$
\det(A)\neq 0.
$$

\section{Układ równań liniowych}

Układ
$$
\begin{cases}
a_{11}x_1+\cdots+a_{1n}x_n=b_1,\\
\vdots\\
a_{n1}x_1+\cdots+a_{nn}x_n=b_n
\end{cases}
\qquad\Longleftrightarrow\qquad
Ax=b,
$$

z
$$
A\in\mathbb R^{n\times n},\quad x\in\mathbb R^n,\quad b\in\mathbb R^n.
$$

\textbf{Fakt.}
Układ $Ax=b$ ma dokładnie jedno rozwiązanie wtedy i tylko wtedy, gdy $\det(A)\neq 0$.

Dwie klasyczne postacie rozwiązania:
\begin{itemize}
\item przez macierz odwrotną:
$$
x=A^{-1}b,
$$
\item wzory Cramera:
$$
x_k=\frac{\det(A_k)}{\det(A)},\qquad k=1,\dots,n,
$$
gdzie $A_k$ powstaje z $A$ przez zastąpienie $k$-tej kolumny wektorem $b$.
\end{itemize}

\section{Uwaga o uwarunkowaniu}

Nawet mała perturbacja danych może silnie zmienić rozwiązanie.
Na tablicy był przykład dla macierzy bliskiej osobliwej:
$$
A=\begin{bmatrix}1 & 0.99\\ 0.99 & 0.98\end{bmatrix},
$$
porównujący rozwiązania układów $Ax=b$ oraz $A\tilde x=\tilde b$.

\textbf{Uwaga.}
Część wartości liczbowych perturbacji na zdjęciu jest nieczytelna; sens przykładu: zadanie jest źle uwarunkowane.

\section{Układy trójkątne}

\textbf{Układ dolnotrójkątny:}
$$
Lx=b,
\qquad
L=[\ell_{ij}],\ \ell_{ij}=0\ \text{dla}\ j>i.
$$
Rozwiązujemy podstawianiem w przód:
$$
x_i=\frac{b_i-\sum_{j=1}^{i-1}\ell_{ij}x_j}{\ell_{ii}},
\qquad i=1,\dots,n,
$$
przy $\ell_{ii}\neq 0$. Koszt: $O(n^2)$.

\textbf{Układ górnotrójkątny:}
$$
Ux=b,
\qquad
U=[u_{ij}],\ u_{ij}=0\ \text{dla}\ j<i.
$$
Rozwiązujemy podstawianiem wstecz:
$$
x_i=\frac{b_i-\sum_{j=i+1}^{n}u_{ij}x_j}{u_{ii}},
\qquad i=n,n-1,\dots,1,
$$
przy $u_{ii}\neq 0$. Koszt: $O(n^2)$.

\section{Metoda faktoryzacji (LU)}

Niech $A$ będzie odwracalna i załóżmy, że
$$
A=LU,
$$
gdzie $L$ jest trójkątna dolna, a $U$ trójkątna górna.

Wtedy
$$
Ax=b
\iff
(LU)x=b
\iff
L(Ux)=b.
$$
Wprowadzamy zmienną pomocniczą $y=Ux$ i rozwiązujemy dwa układy:
$$
\begin{cases}
Ly=b,\\
Ux=y.
\end{cases}
$$

\textbf{Korzyść.}
Zamiast pełnego układu rozwiązujemy dwa układy trójkątne.

Koszt:
$$
O(n^3)\ \text{(wyznaczenie }L,U\text{)}+O(n^2)+O(n^2)=O(n^3).
$$

\section{Przykład}

Rozwiązać $Ax=b$ metodą LU, gdzie
$$
A=\begin{bmatrix}
1&2&3\\
-3&-2&-4\\
-5&18&26
\end{bmatrix},
\qquad
b=\begin{bmatrix}14\\ -19\\ 109\end{bmatrix}.
$$

Dla
$$
L=\begin{bmatrix}
1&0&0\\
-3&1&0\\
-5&7&1
\end{bmatrix},
\qquad
U=\begin{bmatrix}
1&2&3\\
0&4&5\\
0&0&6
\end{bmatrix}
$$
mamy $A=LU$.

1) Rozwiązujemy $Ly=b$:
$$
y=\begin{bmatrix}14\\ 23\\ 18\end{bmatrix}.
$$

2) Rozwiązujemy $Ux=y$:
$$
x=\begin{bmatrix}1\\ 2\\ 3\end{bmatrix}.
$$

\section{Warunek istnienia rozkładu LU (bez pivotingu)}

\textbf{Twierdzenie.}
Jeśli wszystkie minory główne wiodące macierzy $A=[a_{ij}]\in\mathbb R^{n\times n}$ są niezerowe,
$$
\det\!
\begin{bmatrix}
a_{11} & \cdots & a_{1k}\\
\vdots & \ddots & \vdots\\
a_{k1} & \cdots & a_{kk}
\end{bmatrix}
\neq 0,
\qquad k=1,2,\dots,n,
$$
to istnieje dokładnie jeden rozkład
$$
A=LU,
$$
gdzie $L$ jest dolnotrójkątna z jedynkami na przekątnej, a $U$ jest górnotrójkątna.

Współczynniki wyznaczamy rekurencyjnie:
$$
u_{ij}=a_{ij}-\sum_{k=1}^{i-1}\ell_{ik}u_{kj},\qquad i\le j,
$$
$$
\ell_{ij}=\frac{a_{ij}-\sum_{k=1}^{j-1}\ell_{ik}u_{kj}}{u_{jj}},\qquad i>j.
$$

\section{Zastosowania rozkładu LU}

\begin{itemize}
\item Obliczanie wyznacznika:
$$
\det(A)=\det(L)\det(U)=u_{11}u_{22}\cdots u_{nn}
$$
(bo w tym wariancie $\det(L)=1$).
\item Obliczanie macierzy odwrotnej:
$$
A^{-1}=(LU)^{-1}=U^{-1}L^{-1},
$$
co sprowadza się do rozwiązywania układów trójkątnych.
\end{itemize}

\textbf{Uwaga końcowa.}
Na tablicy pojawia się zapowiedź alternatywnego wariantu rozkładu (inna normalizacja elementów diagonalnych); temat przeniesiony na kolejny wykład.
