\textbf{Powtórka (z wykładu 13).}
\begin{itemize}
\item Rozkład $A=LU$ pozwala rozwiązywać układ $Ax=b$ przez dwa układy trójkątne:
$$
Ly=b,\qquad Ux=y.
$$
\item Koszt: wyznaczenie $L,U$ rzędu $O(n^3)$, a każde podstawianie trójkątne rzędu $O(n^2)$.
\item Z rozkładu $LU$:
$$
\det(A)=\det(L)\det(U),
\qquad
A^{-1}=(LU)^{-1}=U^{-1}L^{-1}.
$$
\end{itemize}

\section{Odwracanie macierzy przez układy z wektorami bazowymi}

Niech $A\in\mathbb R^{n\times n}$, $\det(A)\neq 0$, oraz
$$
X:=A^{-1}=[x_1\ \cdots\ x_n].
$$
Z równania
$$
AX=I_n=[e_1\ \cdots\ e_n]
$$
dostajemy niezależne układy:
$$
Ax_1=e_1,\quad Ax_2=e_2,\ \dots,\ Ax_n=e_n.
$$

Jeśli mamy już rozkład $A=LU$, to dla każdego $k$ rozwiązujemy:
$$
Ly_k=e_k,\qquad Ux_k=y_k.
$$

Koszt całkowity:
$$
O(n^3)+n\cdot O(n^2)=O(n^3).
$$

\textbf{Ważna uwaga numeryczna.}
W praktyce numerycznej zwykle unikamy jawnego wyznaczania $A^{-1}$, gdy celem jest rozwiązanie $Ax=b$.

\section{Eliminacja Gaussa -- schemat}

Rozważamy układ
$$
A^{(1)}x=b^{(1)},
$$
gdzie standardowo $A^{(1)}=A$, $b^{(1)}=b$.

\subsection*{Krok 1}
Zakładamy $a_{11}^{(1)}\neq 0$ i dla $i=2,3,\dots,n$ bierzemy mnożniki
$$
m_{i1}:=-\frac{a_{i1}^{(1)}}{a_{11}^{(1)}}.
$$
Dodając $m_{i1}$-krotność pierwszego równania do $i$-tego, dostajemy:
$$
a_{ij}^{(2)}=a_{ij}^{(1)}+m_{i1}a_{1j}^{(1)},
\qquad j=2,3,\dots,n,
$$
$$
b_i^{(2)}=b_i^{(1)}+m_{i1}b_1^{(1)},
\qquad i=2,3,\dots,n.
$$

\subsection*{Krok ogólny}
Po $(r-1)$ krokach mamy układ
$$
A^{(r)}x=b^{(r)},
$$
i eliminujemy zmienną $x_{r-1}$ z równań $i=r,\dots,n$.
Zakładając $a_{r-1,r-1}^{(r-1)}\neq 0$, bierzemy
$$
m_{i,r-1}:=-\frac{a_{i,r-1}^{(r-1)}}{a_{r-1,r-1}^{(r-1)}},
\qquad i=r,\dots,n,
$$
po czym aktualizujemy
$$
a_{ij}^{(r)}=a_{ij}^{(r-1)}+m_{i,r-1}a_{r-1,j}^{(r-1)},
\qquad i,j=r,\dots,n,
$$
$$
b_i^{(r)}=b_i^{(r-1)}+m_{i,r-1}b_{r-1}^{(r-1)},
\qquad i=r,\dots,n.
$$

Po $n-1$ krokach otrzymujemy układ trójkątny górny:
$$
\sum_{j=r}^{n}a_{rj}^{(n)}x_j=b_r^{(n)},
\qquad r=1,2,\dots,n,
$$
z elementami głównymi $a_{rr}^{(r)}\neq 0$.

\section{Podstawianie wsteczne i koszt}

Rozwiązanie układu trójkątnego górnego:
$$
x_r=\frac{b_r^{(n)}-\sum_{j=r+1}^{n}a_{rj}^{(n)}x_j}{a_{rr}^{(r)}},
\qquad r=n,n-1,\dots,1.
$$

Koszt:
\begin{itemize}
\item faza eliminacji: $\sim O\!\left(\frac{n^3}{3}\right)$,
\item podstawianie wsteczne: $O(n^2)$,
\item łącznie: $O(n^3)$.
\end{itemize}

\section{Problem numeryczny: zanik elementu głównego}

Może się zdarzyć, że mimo istnienia jednoznacznego rozwiązania, w pewnym kroku eliminacji dostajemy
$$
a_{rr}^{(r)}=0
$$
lub bardzo małą wartość, co psuje obliczenia (dzielenie przez zero albo duże błędy zaokrągleń).

Aby temu zapobiec, stosujemy \textbf{wybór elementów głównych} (pivoting).

\section{Wybór elementów głównych}

\textbf{1) Częściowy wybór elementów głównych (partial pivoting).}
W kroku $r$ zamieniamy miejscami wiersz $r$ z wierszem $p$, gdzie
$$
|a_{pr}^{(r)}|=\max_{r\le i\le n}|a_{ir}^{(r)}|.
$$

\textbf{2) Pełny wybór elementów głównych (full pivoting).}
W kroku $r$ wybieramy $(p,q)$ takie, że
$$
|a_{pq}^{(r)}|=\max_{r\le i,j\le n}|a_{ij}^{(r)}|,
$$
a następnie zamieniamy zarówno wiersze, jak i kolumny (czyli także numerację niewiadomych).

\section{Wniosek o rozkładzie z permutacją}

\textbf{Twierdzenie.}
Dla dowolnej macierzy $A\in\mathbb R^{n\times n}$ istnieją:
\begin{itemize}
\item macierz permutacji wierszy $P$,
\item macierz dolnotrójkątna $L$ z jedynkami na przekątnej,
\item macierz górnotrójkątna $U$,
\end{itemize}
takie, że
$$
PA=LU.
$$

To jest algebraiczny zapis eliminacji Gaussa z częściowym wyborem elementów głównych.
