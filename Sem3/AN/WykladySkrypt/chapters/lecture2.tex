\section{Błąd bezwzględny i bład względny}

Weźmy liczby:
$$x = 1.23456789 \quad \tilde{x} = 1.2345679$$
$$y = 10^{50}+1 \quad \tilde{y} = 10^{50}$$

Wtedy błąd bezwzględny to:
$$|x-\tilde{x}| = 10^{-8} \quad |y-\tilde{y}| = 1$$

Błąd względny to:
$$\frac{|x-\tilde{x}|}{|x|} = 0.8 \cdot 10^{-8} \quad \frac{|y-\tilde{y}|}{|y|} = 10^{-50}$$

Błąd względny jest lepszą miarą błędu.

Dodatkowo zdefiniujmy liczbę cyfr dokładnych:
$$acc(v, \tilde{v}) = -\log_{10} \left( \middle|1 - \frac{\tilde{v}}{v} \middle| \right)$$

Wtedy:
$$acc(x, \tilde{x}) \approx 8.091 \quad acc(y, \tilde{y}) = 50$$


\section{Reprezentacja liczb w komputerze}
Potrafimy reprezentować wszystkie liczby całkowite z pewnego zakresu, ale nie potrafimy reprezentować wszystkich liczb rzeczywistych. Dlatego musimy wybrać pewną reprezentację, która będzie przybliżać liczby rzeczywiste.

a) $l \in \mathbb{Z}$,
$$l = \pm \sum_{i=0}^{n} e_i 2^i, \quad e_i \in \{0,1\}, \quad e_n = 1$$

Jeśli $n < d$ to OK, a jeśli $n \ge d$ to przepełnienie.

\vspace{0.5cm}

b) $x \in \mathbb{R}\setminus\{0\}$

\textbf{TW.} Dla każdej liczby rzeczywistej $x \neq 0$ istnieje trójka:
$$m \in \left[\frac{1}{2},1\right) \quad \text{(mantysa)}$$
$$c \in \mathbb{Z} \quad \text{(cecha)}$$
$$s \in \{+1,-1\} \quad \text{(znak liczby)}$$
dla których
$$x = s \cdot m \cdot 2^c$$
Trójka $(s,m,c)$ jest wyznaczona jednoznacznie.

