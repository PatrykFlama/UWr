
\textbf{Powtórka z wykładu 4.}
\begin{itemize}
\item Bisekcja: rząd zbieżności $p=1$.
\item Newton: rząd zbieżności $p=2$ (lokalnie, przy spełnionych założeniach).
\item Sieczne: rząd zbieżności $p\approx 1.618$.
\end{itemize}

\section{Postacie wielomianu}

Oznaczenie:
$$\Pi_n:=\{w\colon w \text{ jest wielomianem stopnia } \le n\}.$$

\textbf{a) Postać naturalna (potęgowa).}
Dla $w\in\Pi_n$:
$$w(x)=\sum_{i=0}^{n} a_i x^i.$$

\textbf{Schemat Hornera.}
\[
w(x)=(((a_nx+a_{n-1})x+a_{n-2})x+\dots+a_1)x+a_0.
\]

Algorytm:
\[
\begin{array}{l}
w_n:=a_n,\\
w_k:=w_{k+1}x+a_k,\quad k=n-1,n-2,\dots,0.
\end{array}
\]
Wtedy $w(x)=w_0$. Koszt obliczeniowy: $O(n)$.

\vspace{0.3cm}

\textbf{b) Postać Newtona.}
Niech $x_0,x_1,\dots,x_n$ będą ustalonymi punktami. Wtedy
$$
w(x)=\sum_{k=0}^{n} b_k P_k(x),
$$
gdzie
$$
P_0(x)=1,\qquad
P_k(x)=\prod_{i=0}^{k-1}(x-x_i)\quad (k\ge 1).
$$

Czyli:
$$
w(x)=b_0+b_1(x-x_0)+b_2(x-x_0)(x-x_1)+\dots+b_n\prod_{i=0}^{n-1}(x-x_i).
$$

\textbf{Uogólniony schemat Hornera (dla postaci Newtona).}
\[
\begin{array}{l}
w_n:=b_n,\\
w_k:=w_{k+1}(x-x_k)+b_k,\quad k=n-1,n-2,\dots,0.
\end{array}
\]
Wtedy $w(x)=w_0$.

\vspace{0.3cm}

\textbf{c) Postać Czebyszewa.}
Wielomiany Czebyszewa definiujemy rekurencyjnie:
$$T_0(x)=1,\qquad T_1(x)=x,\qquad T_k(x)=2xT_{k-1}(x)-T_{k-2}(x)\quad (k\ge 2).$$

\textbf{Własności (podstawowe).}
\begin{itemize}
\item $T_k\in \Pi_k\setminus \Pi_{k-1}$.
\item Dla $n\ge 1$: $T_n(x)=2^{n-1}x^n+\dots$
\item Parzystość: $T_n(-x)=(-1)^nT_n(x)$.
\item Dla $x\in[-1,1]$: $T_n(x)=\cos\!\big(n\arccos(x)\big)$.
\item Wszystkie miejsca zerowe $T_n$ są rzeczywiste, pojedyncze i należą do $(-1,1)$.
\item Dla $x\in[-1,1]$: $|T_n(x)|\le 1$.
\item $\operatorname{lin}\{T_0,T_1,\dots,T_n\}=\Pi_n$.
\end{itemize}

Postać Czebyszewa wielomianu:
$$
w(x)=\frac12 c_0T_0(x)+c_1T_1(x)+\dots+c_nT_n(x)
=\sum_{k=0}^{n}{}' c_kT_k(x),
$$
gdzie kreska przy sumie oznacza, że składnik dla $k=0$ jest liczony z czynnikiem $\frac12$.

\textbf{Algorytm Clenshawa} (obliczanie wartości w postaci Czebyszewa, koszt $O(n)$):
\[
\begin{array}{l}
B_{n+2}:=0,\quad B_{n+1}:=0,\\
B_k:=2xB_{k+1}-B_{k+2}+c_k,\quad k=n,n-1,\dots,0.
\end{array}
\]
Wtedy
$$
w(x)=\frac{B_0-B_2}{2}.
$$

\section{Interpolacja wielomianowa}

Mamy dane pomiary:
$$
x_0,x_1,\dots,x_n\in\mathbb{R},\quad x_i\neq x_j\ (i\neq j),
\qquad
y_0,y_1,\dots,y_n\in\mathbb{R}.
$$

Szukamy wielomianu $L_n\in\Pi_n$ takiego, że
$$
L_n(x_k)=y_k,\qquad k=0,1,\dots,n.
$$

\begin{center}
\begin{tikzpicture}[x=1.2cm,y=1cm,>=Latex]
  \draw[->] (-0.2,0) -- (6.5,0) node[right] {$x$};
  \draw[->] (0,-0.2) -- (0,3.2) node[above] {$y$};
  \draw[thick,blue] plot[smooth] coordinates {(0.4,0.6) (1.2,2.0) (2.2,1.3) (3.1,2.4) (4.3,1.2) (5.8,2.2)};
  \foreach \xx/\yy/\lab in {0.4/0.6/0,1.2/2.0/1,2.2/1.3/2,3.1/2.4/3,4.3/1.2/4,5.8/2.2/5} {
    \fill (\xx,\yy) circle (1.3pt);
    \draw[densely dashed,gray] (\xx,0)--(\xx,\yy);
    \node[below] at (\xx,0) {$x_{\lab}$};
  }
  \node at (4.9,2.8) {$L_n(x)$};
\end{tikzpicture}
\end{center}

\textbf{Twierdzenie.}
Zadanie interpolacyjne Lagrange'a ma zawsze dokładnie jedno rozwiązanie.

\textbf{Postać Lagrange'a.}
$$
L_n(x)=\sum_{k=0}^{n}\lambda_k(x)\,y_k,
$$
gdzie
$$
\lambda_k(x):=\prod_{\substack{i=0\\ i\neq k}}^{n}\frac{x-x_i}{x_k-x_i},\qquad k=0,1,\dots,n.
$$

Własność bazowa:
$$
\lambda_k(x_i)=
\begin{cases}
1,& i=k,\\
0,& i\neq k.
\end{cases}
$$

\textbf{Przykład.}
Niech $f(x)=e^x$, $n=3$, węzły:
$$x_0=0,\quad x_1=0.2,\quad x_2=0.6,\quad x_3=0.8.$$

Budujemy $L_3$ i przybliżamy $f(0.4)$.
W zapisie z tablicy:
$$
L_3(x)=\lambda_0(x)y_0+\lambda_1(x)y_1+\lambda_2(x)y_2+\lambda_3(x)y_3,\qquad y_k=f(x_k).
$$
Otrzymane przybliżenie:
$$
L_3(0.4)\approx 1.49142,\qquad f(0.4)\approx 1.49182.
$$
Zatem błąd punktowy jest rzędu $10^{-3}$.

\vspace{0.3cm}

\textbf{Przykłady}
\begin{itemize}
\item $f(x)=\sin(x)$, \quad
$$x_k=\frac{2k\pi}{n},\qquad 0\le k\le n,\qquad n=1,\dots,15,$$
i porównujemy z interpolantem $L_n(x)$.

\item $f(x)=|x|$, \quad
$$x_k=-1+\frac{2k}{n},\qquad 0\le k\le n,\qquad n=1,\dots,15,$$
i obserwujemy jakość przybliżenia przez $L_n(x)$.

\item $f(x)=x^6$, \quad
$$x_k=-2+\frac{4k}{n},\qquad 0\le k\le n,\qquad n=1,\dots,6.$$
Ponieważ $f\in\Pi_6$, dla $n\ge 6$ mamy dokładnie
$$L_n(x)=x^6 \quad (\text{w szczególności } L_6(x)=x^6).$$
\end{itemize}
