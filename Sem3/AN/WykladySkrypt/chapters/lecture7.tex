
\textbf{Powtórka.}
\begin{itemize}
\item Postać Newtona + ilorazy różnicowe: budowa w $O(n^2)$.
\item Obliczanie wartości wielomianu Newtona: $O(n)$ (uogólniony schemat Hornera).
\item Błąd interpolacji:
$$
|f(x)-L_n(x)|\le
\max_{t\in[a,b]}\left|\frac{f^{(n+1)}(t)}{(n+1)!}\right|
\cdot
\max_{x\in[a,b]}\left|\prod_{k=0}^{n}(x-x_k)\right|.
$$
\item Dobór „dobrych” węzłów: węzły Czebyszewa.
\end{itemize}

\section{Definicja NIFS3}

Niech
$$
a=t_0<t_1<\dots<t_n=b,\qquad y_k=f(t_k).
$$

Funkcję $s$ nazywamy \textbf{NIFS3} (naturalną interpolacyjną funkcją sklejaną stopnia 3), jeśli:
\begin{itemize}
\item $s(t_k)=y_k$ dla $k=0,\dots,n$,
\item na każdym $[t_{k-1},t_k]$ funkcja $s$ jest wielomianem stopnia co najwyżej 3,
\item $s,s',s''\in C[a,b]$,
\item warunki naturalności: $s''(a)=s''(b)=0$.
\end{itemize}

\begin{center}
\begin{tikzpicture}[x=1.0cm,y=1.0cm,>=Latex]
  \draw[->] (-0.2,0) -- (8.2,0) node[right] {$t$};
  \draw[->] (0,-0.2) -- (0,3.0);
  \draw[thick,blue] plot[smooth] coordinates {(0.2,1.2) (1.8,1.7) (3.4,1.3) (5.0,1.9) (6.6,1.7) (7.8,2.4)};
  \foreach \x/\y/\k in {0.2/1.2/0,1.8/1.7/1,3.4/1.3/2,5.0/1.9/3,6.6/1.7/4,7.8/2.4/5} {
    \fill[red] (\x,\y) circle (1.2pt);
    \draw[densely dashed,gray] (\x,0) -- (\x,\y);
    \node[below] at (\x,0) {$t_{\k}$};
  }
\end{tikzpicture}
\end{center}

\textbf{Ile warunków?}
\begin{itemize}
\item interpolacja w węzłach: $n+1$,
\item ciągłość $s'$ i $s''$ w węzłach wewnętrznych: $2(n-1)$,
\item naturalność: $2$.
\end{itemize}
Razem: $4n$ warunków (tyle co współczynników dla $n$ wielomianów kubicznych).

\section{Postać klamrowa i przykład}

Na przedziale $[t_{k-1},t_k]$:
$$
s(t)=s_k(t)=A_kt^3+B_kt^2+C_kt+D_k.
$$

\textbf{Przykład:}
dla danych $(t_0,y_0)=(-1,1)$, $(t_1,y_1)=(0,-1)$, $(t_2,y_2)=(1,1)$ otrzymujemy
$$
s(t)=
\begin{cases}
t^3+3t^2-1,& t\in[-1,0],\\
-t^3+3t^2-1,& t\in[0,1].
\end{cases}
$$

\textbf{Wniosek.}
Bezpośrednie rozwiązywanie pełnego układu dla współczynników klamrowych jest mało efektywne dla dużych $n$.

\section{Momenty i układ trójprzekątniowy}

Wprowadzamy momenty:
$$
M_k:=s''(t_k),\qquad k=0,1,\dots,n,\qquad M_0=M_n=0.
$$

Niech
$$
h_k:=t_k-t_{k-1},\qquad
\lambda_k:=\frac{h_k}{h_k+h_{k+1}},\qquad
d_k:=6\,f[t_{k-1},t_k,t_{k+1}],\quad k=1,\dots,n-1.
$$

Momenty spełniają układ:
$$
\lambda_k M_{k-1}+2M_k+(1-\lambda_k)M_{k+1}=d_k,\qquad k=1,\dots,n-1.
$$
Jest to układ liniowy trójprzekątniowy, więc można go rozwiązać w czasie $O(n)$.

\textbf{Uwaga.} Na egzaminie nie używamy bezpośrednio gotowego wzoru jawnego na segment $s_k(t)$.

\section{Algorytm O(n)}

Obliczamy pomocnicze wielkości rekurencyjnie:
$$
q_0:=0,\qquad u_0:=0,
$$
$$
\begin{aligned}
p_k&:=\lambda_k q_{k-1}+2,\\
q_k&:=\frac{\lambda_k-1}{p_k},\\
u_k&:=\frac{d_k-\lambda_k u_{k-1}}{p_k},
\end{aligned}
\qquad k=1,2,\dots,n-1.
$$

Następnie:
$$
M_{n-1}=u_{n-1},\qquad
M_k=u_k+q_k M_{k+1},\quad k=n-2,n-3,\dots,1.
$$

Po wyznaczeniu momentów rekonstruujemy każdy segment splajnu.

\section{Istnienie i jednoznaczność}

\textbf{Twierdzenie.}
Dla dowolnych danych
$$
n\in\mathbb{N},\quad a=t_0<t_1<\dots<t_n=b,\quad y_k=f(t_k),
$$
istnieje dokładnie jedna naturalna interpolacyjna funkcja sklejona 3-go stopnia $s$ spełniająca
$$
s''(a)=s''(b)=0.
$$

\section{Zastosowanie w grafice komputerowej}

W grafice komputerowej często buduje się krzywą parametryczną
$$
\gamma(t)=(x(t),y(t)),\qquad t\in[t_0,t_n],
$$
gdzie $x(t)$ i $y(t)$ są niezależnymi NIFS3 budowanymi dla tych samych parametrów $t_k$.

Praktyczny schemat:
\begin{itemize}
\item wybieramy punkty kontrolne $(x_k,y_k)$,
\item dobieramy parametry $t_k$ (często prawie równoległe długości łuków),
\item interpolujemy osobno:
$$
s_x(t_k)=x_k,\qquad s_y(t_k)=y_k,
$$
\item otrzymujemy krzywą
$$
\gamma(t)=\big(s_x(t),s_y(t)\big).
$$
\end{itemize}

\begin{center}
\begin{tikzpicture}[x=1.0cm,y=1.0cm,>=Latex]
  \draw[->] (-0.2,0) -- (8.2,0) node[right] {$x$};
  \draw[->] (0,-0.2) -- (0,3.6) node[above] {$y$};
  \draw[thick,blue] plot[smooth] coordinates {(0.4,0.8) (1.6,2.6) (2.7,1.9) (3.8,0.9) (5.1,2.7) (6.6,1.4) (7.6,2.5)};
  \foreach \x/\y in {0.4/0.8,1.6/2.6,2.7/1.9,3.8/0.9,5.1/2.7,6.6/1.4,7.6/2.5} {
    \fill[red] (\x,\y) circle (1.2pt);
  }
  \node at (6.4,3.1) {$\gamma(t)$};
\end{tikzpicture}
\end{center}
